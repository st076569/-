% <============================ Древесные структуры данных ============================>
% Баталов Семен, группа 151, 2020
% Конспект по программированию
% <============================ Древесные структуры данных ============================>

\documentclass[12pt, a4paper]{report}			% Размер бумаги устанавливаем А4, шрифт 12пунктов
\usepackage{graphicx}						% Подключаем графический модуль
\usepackage{multirow}						% Возможность объединять строки таблицы
\graphicspath{{../../pictures/}}					% Путь к картинкам
\usepackage[utf8]{inputenc}					% Включаем свою кодировку: utf8
\usepackage[english, russian]{babel}			% Используем русский и английский языки с переносами
\usepackage{indentfirst}						% Красная строка для первого параграфа
\usepackage{misccorr}						% Соответствие стандарту
\usepackage{tikz}								% Подключаем модуль диаграмм
\usepackage{amsmath}						% Математика
\usepackage{listingsutf8}

\begin{document}

\title{Древесные структуры данных}
\author{Баталов Семен}
\date{2 мая 2020 г.}
\maketitle

\tableofcontents
\newpage

\section{Предисловие}

В данном разделе будут рассматриваться бинарные деревья. Будут освещены некоторые алгоритмы работы с такими структурами данных 
и асимптотические оценки этих алгоритмов. Так же будут описаны свойства деревьев и разновидности этих структур даннных. 

\section{Основная терминология}

Ниже представлено бинарное дерево (дерево, в котором каждый узел имеет не более двух потомков). Справа от него расположено 
количество элементов на каждом уровне и итоговое количество элементов дерева.

Стоит отметить, что среднее время поиска одного элемента $O(n),$ где~$n = 2^{k + 1} - 1.$ При этом среднее время вставки логарифмическое~$O(\log n)$.

% <<<<Бинарное дерево>>>>
\begin{center}
	\begin{tikzpicture}[level/.style={sibling distance=50mm/#1}]
	\node {$Root_{0}$}
		child{node {$Cell_{1}$}
			child{node {$Cell_{2}$}
				child{node {$\vdots$}
					child{node {$Leaf_{k}$}}
					child{node {$Leaf_{k}$}}
				}
				child{node {$\vdots$}}
			}
			child{node {$Cell_{2}$}
				child{node {$\vdots$}}
				child{node {$\vdots$}}
			}
		}
		child{node {$Cell_{1}$}
			child{node {$Cell_{2}$}
				child{node {$\vdots$}}
				child{node {$\vdots$}}
			}
			child{node {$Cell_{2}$}
				child{node {$\vdots$}}
				child{node {$\vdots$}
					child{node {$Leaf_{k}$}}
					child{node {$Leaf_{k}$}
						child [grow=right] {node (a) {$2^{k}$} edge from parent[draw=none]
							child [grow=up] {node (b) {$\vdots$} edge from parent[draw=none]
								child [grow=up] {node (c) {$2^{2}$} edge from parent[draw=none]
									child [grow=up] {node (d) {$2^{1}$} edge from parent[draw=none]
										child [grow=up] {node (e) {$2^{0}$} edge from parent[draw=none]}
									}
								}
							}
							child [grow=down] {node (f) {($2^{k+1} - 1)$} edge from parent[draw=none]}
						}
					}
				}
			}
		};
	\path (a) -- (b) node [midway] {+};
	\path (b) -- (c) node [midway] {+};
	\path (c) -- (d) node [midway] {+};
	\path (d) -- (e) node [midway] {+};
	\path (a) -- (f) node [midway] {\textbardbl};
	\end{tikzpicture}
\end{center}
% <<<<Бинарное дерево>>>>

\subsection{Определения}
\begin{itemize}
	\item \textbf{Потомки}~--~это те элементы, в которые можем попасть из текущего.
	\item \textbf{Предки}~--~это те элементы, из которых можем попасть в текущий.
	\item \textbf{Корень}~--~это элемент дерева без предков.
	\item \textbf{Лист}~--~это элемент дерева без потомков.
	\item \textbf{Прямой потомок}~--~это тот потомок, в который попадаем за один шаг.
	\item \textbf{Прямой предок}~--~это тот предок, из которого попадаем за один шаг.
	\item \textbf{Глубина элемента}~--~это расстояние от корня до текущего элемента.
	\item \textbf{Высота дерева}~--~это максимальная глубина.
	\item \textbf{Уровень}~--~это множество элементов с одинаковой глубиной.
	\item \textbf{Идеально сбалансированное дерево}~--~это дерево, в котором у всех элементов кроме элементов последних 
	двух уровней есть ровно два потомка.
	\item \textbf{Сбалансированное дерево}~--~это дерево, высота которого логарифмична количеству элементов.
\end{itemize}

\section{Дерево поиска}

Рассмотрим бинарное дерево, где каждому элементу приписали ключ~(некоторое целое число), уникальный 
для каждого элемента.

\subsection{Определение}
\begin{itemize}
	\item \textbf{Дерево поиска}~--~это дерево, в котором для каждого элемента выполнено свойство: Значение ключа всех 
	левых потомков меньше значения ключа текущего, а значение ключей всех правых потомков больше текущего.
\end{itemize}

Ниже приведен пример дерева поиска. За счет упорядоченной структуры это дерево позволяет эффективно решать задачу поиска.
Далее мы рассмотрим всевозможные способы обхода этого дерева.

% <<<<Дерево поиска>>>>
\begin{center}
	\begin{tikzpicture}[sibling distance=5cm, level distance=1.7cm]
	\node [circle, minimum size=1.2cm, draw] {\large $9$}
		child {node [circle, minimum size=1.2cm, draw] {\large $6$}
			child [grow=south west] {node [circle, minimum size=1.2cm, draw] {\large $3$}
				child [grow=south west] {node [circle, minimum size=1.2cm, draw] {\large $2$}}
				child [grow=south east] {node [circle, minimum size=1.2cm, draw] {\large $4$}}
			}
			child [grow=south east] {node [circle, minimum size=1.2cm, draw] {\large $8$}}
		}
		child {node [circle, minimum size=1.2cm, draw] {\large $10$}
			child [grow=south west] {node [circle, minimum size=1.2cm, draw] {\large $10$}}
			child [grow=south east] {node [circle, minimum size=1.2cm, draw] {\large $15$}
				child [grow=south west] {node [circle, minimum size=1.2cm, draw] {\large $11$}}
			}
		};
	\end{tikzpicture}
\end{center}
% <<<<Дерево поиска>>>>

\subsection{Обход в глубину (prefix)}

\begin{enumerate}
	\item Обрабатываем элемент ($O(1)$)
	\item Рекурсивно переходим к левому потомку ($O(1)$)
	\item Рекурсивно переходим к правому потомку ($O(1)$)
\end{enumerate}

Далее приведен пример такого обхода в глубину. Нижний индекс указывает, каким по счету будет обрабатываться элемент.
Общая оценка сложности обхода в глубину $O(n)$.

% <<<<Обход в глубину (prefix)>>>>
\begin{center}
	\begin{tikzpicture}[sibling distance=5cm, level distance=1.7cm]
	\node [circle, minimum size=1.2cm, draw] {\large $9_{1}$}
		child {node [circle, minimum size=1.2cm, draw] {\large $6_{2}$}
			child [grow=south west] {node [circle, minimum size=1.2cm, draw] {\large $3_{3}$}
				child [grow=south west] {node [circle, minimum size=1.2cm, draw] {\large $2_{4}$}}
				child [grow=south east] {node [circle, minimum size=1.2cm, draw] {\large $4_{5}$}}
			}
			child [grow=south east] {node [circle, minimum size=1.2cm, draw] {\large $8_{6}$}}
		}
		child {node [circle, minimum size=1.2cm, draw] {\large $10_{7}$}
			child [grow=south west] {node [circle, minimum size=1.2cm, draw] {\large $10_{8}$}}
			child [grow=south east] {node [circle, minimum size=1.2cm, draw] {\large $15_{9}$}
				child [grow=south west] {node [circle, minimum size=1.2cm, draw] {\large $11_{10}$}}
			}
		};
	\end{tikzpicture}
\end{center}
% <<<<Обход в глубину (prefix)>>>>

\subsection{Обход в глубину (postfix)}

\begin{enumerate}
	\item Рекурсивно переходим к левому потомку
	\item Рекурсивно переходим к правому потомку
	\item Обрабатываем элемент
\end{enumerate}

Далее приведен пример такого обхода в глубину. Нижний индекс указывает, каким по счету будет обрабатываться элемент.
Общая оценка сложности обхода в глубину $O(n)$.

% <<<<Обход в глубину (postfix)>>>>
\begin{center}
	\begin{tikzpicture}[sibling distance=5cm, level distance=1.7cm]
	\node [circle, minimum size=1.2cm, draw] {\large $9_{10}$}
		child {node [circle, minimum size=1.2cm, draw] {\large $6_{5}$}
			child [grow=south west] {node [circle, minimum size=1.2cm, draw] {\large $3_{3}$}
				child [grow=south west] {node [circle, minimum size=1.2cm, draw] {\large $2_{1}$}}
				child [grow=south east] {node [circle, minimum size=1.2cm, draw] {\large $4_{2}$}}
			}
			child [grow=south east] {node [circle, minimum size=1.2cm, draw] {\large $8_{4}$}}
		}
		child {node [circle, minimum size=1.2cm, draw] {\large $10_{9}$}
			child [grow=south west] {node [circle, minimum size=1.2cm, draw] {\large $10_{6}$}}
			child [grow=south east] {node [circle, minimum size=1.2cm, draw] {\large $15_{8}$}
				child [grow=south west] {node [circle, minimum size=1.2cm, draw] {\large $11_{7}$}}
			}
		};
	\end{tikzpicture}
\end{center}
% <<<<Обход в глубину (postfix)>>>>

\subsection{Обход в глубину (infix)}

\begin{enumerate}
	\item Рекурсивно переходим к левому потомку
	\item Обрабатываем элемент
	\item Рекурсивно переходим к правому потомку
\end{enumerate}

Далее приведен пример такого обхода в глубину. Нижний индекс указывает, каким по счету будет обрабатываться элемент.

% <<<<Обход в глубину (infix)>>>>
\begin{center}
	\begin{tikzpicture}[sibling distance=5cm, level distance=1.7cm]
	\node [circle, minimum size=1.2cm, draw] {\large $9_{6}$}
		child {node [circle, minimum size=1.2cm, draw] {\large $6_{4}$}
			child [grow=south west] {node [circle, minimum size=1.2cm, draw] {\large $3_{2}$}
				child [grow=south west] {node [circle, minimum size=1.2cm, draw] {\large $2_{1}$}}
				child [grow=south east] {node [circle, minimum size=1.2cm, draw] {\large $4_{3}$}}
			}
			child [grow=south east] {node [circle, minimum size=1.2cm, draw] {\large $8_{5}$}}
		}
		child {node [circle, minimum size=1.2cm, draw] {\large $10_{8}$}
			child [grow=south west] {node [circle, minimum size=1.2cm, draw] {\large $10_{7}$}}
			child [grow=south east] {node [circle, minimum size=1.2cm, draw] {\large $15_{10}$}
				child [grow=south west] {node [circle, minimum size=1.2cm, draw] {\large $11_{9}$}}
			}
		};
	\end{tikzpicture}
\end{center}
% <<<<Обход в глубину (infix)>>>>

Данный способ обхода позволяет отсортировать элементы дерева по значению ключа. Причем сложность такой 
сортировки $O(n \cdot \log n)$, так как в дереве всего $n$ элементов, вставка каждого и которых занимает 
$O(\log n)$, а общая оценка сложности обхода $O(n)$.

Объем памяти, требуемый для обхода в глубину: $O(h),$ где~$h$~--~высота дерева.

\subsection{Обход в ширину (bfs)}

Данный обход производится последовательно по каждому уровню. Достаточно составить очередь, в которую будем записывать указатели
на потомки текущего элемента (слева направо). Будем обрабатывать элементы из очереди, пока они не кончатся.

Далее приведен пример такого обхода в ширину. Нижний индекс указывает, каким по счету будет обрабатываться элемент.

% <<<<Обход в ширину>>>>
\begin{center}
	\begin{tikzpicture}[sibling distance=5cm, level distance=1.7cm]
	\node [circle, minimum size=1.2cm, draw] {\large $9_{1}$}
		child {node [circle, minimum size=1.2cm, draw] {\large $6_{2}$}
			child [grow=south west] {node [circle, minimum size=1.2cm, draw] {\large $3_{4}$}
				child [grow=south west] {node [circle, minimum size=1.2cm, draw] {\large $2_{8}$}}
				child [grow=south east] {node [circle, minimum size=1.2cm, draw] {\large $4_{9}$}}
			}
			child [grow=south east] {node [circle, minimum size=1.2cm, draw] {\large $8_{5}$}}
		}
		child {node [circle, minimum size=1.2cm, draw] {\large $10_{3}$}
			child [grow=south west] {node [circle, minimum size=1.2cm, draw] {\large $10_{6}$}}
			child [grow=south east] {node [circle, minimum size=1.2cm, draw] {\large $15_{7}$}
				child [grow=south west] {node [circle, minimum size=1.2cm, draw] {\large $11_{10}$}}
			}
		};
	\end{tikzpicture}
\end{center}
% <<<<Обход в ширину>>>>

Объем памяти, требуемый для обхода в ширину: $O(w),$ где~$w$~--~максимальное количество элементов на уровне.

\section{Самобалансирующиеся деревья}

Здесь мы рассмотрим условия сбалансированности бинарных деревьев поиска и методы их балансировки.

\subsection{Определение}
\begin{itemize}
	\item \textbf{АВЛ дерево}~--~это бинарное дерево поиска, в котором для каждого элемента выполнено свойство: Разница высот 
	левого и правого поддеревьев по модулю не превосходит единицы.
\end{itemize}

Рассмотрим АВЛ дерево высоты $k$. Оценим количество элементов в нем сверху и снизу.

$$
F(k) \le n \le 2^{k+1}-1
$$

В этом неравенстве $F(k) = 1 + F(k - 1) + F(k - 2)$~--~нижняя оценка. При этом $F(k - 1)$ и $F(k - 2)$~--~количество элементов в левом и правом поддеревьях. 
Очевидно $F(k) > Fib(k)$~--~$k$-ое число Фибоначчи.

Из формулы Бине вытекает следующее:

$$
n > \left(\frac{1 + \sqrt 5}{2}\right)^{k}
$$

В итоге получаем оценку сверху и снизу для количества элементов АВЛ дерева:

$$
\varphi^{k} \le n \le \psi^{k}
$$

То есть $n = \theta^{k}$~--~это показательная функция. Следовательно, высота дерева $k = \log n$. По определению 
такое дерево является сбалансированным. То есть АВЛ дерево по определению является сбалансированным.

\section{Вставка и балансировка}

Здесь будут рассматриваться способы балансировки АВЛ дерева при вставке в него элементов. Возьмем некоторое АВЛ дерево: 

% <<<<АВЛ дерево 1>>>>
\begin{center}
	\begin{tikzpicture}[sibling distance=5cm, level distance=1.7cm]
	\node [circle, minimum size=1.2cm, draw] {\large $8$}
		child {node [circle, minimum size=1.2cm, draw] {\large $3$}
			child [grow=south west] {node [circle, minimum size=1.2cm, draw] {\large $1$}}
			child [grow=south east] {node [circle, minimum size=1.2cm, draw] {\large $6$}}
		}
		child {node [circle, minimum size=1.2cm, draw] {\large $14$}
			child [grow=south west] {node [circle, minimum size=1.2cm, draw] {\large $12$}
				child [grow=south west] {node [circle, minimum size=1.2cm, draw] {\large $9$}}
			}
			child [grow=south east] {node [circle, minimum size=1.2cm, draw] {\large $16$}}
		};
	\end{tikzpicture}
\end{center}
% <<<<АВЛ дерево 1>>>>

Попробуем осуществить вставку элемента 10 в него так, как показано на следующем рисунке. 
Дерево перестает быть сбалансированным, так как разность высот поддеревьев вершины 14 по модулю 
становится равной~2.

% <<<<АВЛ дерево 2>>>>
\begin{center}
	\begin{tikzpicture}[sibling distance=5cm, level distance=1.7cm]
	\node [circle, minimum size=1.2cm, draw] {\large $8$}
		child {node [circle, minimum size=1.2cm, draw] {\large $3$}
			child [grow=south west] {node [circle, minimum size=1.2cm, draw] {\large $1$}}
			child [grow=south east] {node [circle, minimum size=1.2cm, draw] {\large $6$}}
		}
		child {node [circle, minimum size=1.2cm, draw] {\large $14$}
			child [grow=south west] {node [circle, minimum size=1.2cm, draw] {\large $12$}
				child [grow=south west] {node [circle, minimum size=1.2cm, draw] {\large $9$}
					child [grow=south east] {node [circle, minimum size=1.2cm, draw] {\large $10$}}
				}
			}
			child [grow=south east] {node [circle, minimum size=1.2cm, draw] {\large $16$}}
		};
	\end{tikzpicture}
\end{center}
% <<<<АВЛ дерево 2>>>>

\subsection{Определение}
\begin{itemize}
	\item \textbf{Поворот дерева}~--~это операция, которая позволяет изменить структуру дерева, не 
	меняя порядка элементов.
\end{itemize}

Чтобы решить проблему разбалансированности дерева поиска, используют следующие повороты:

\begin{enumerate}
	\item Малый левый поворот (SLR)
	\item Малый правый поворот (SRR)
	\item Большой левый поворот (BLR)
	\item Большой правый поворот (BRR)
\end{enumerate}

\subsection{Малый правый и левый повороты}

Рассмотрим следующее АВЛ дерево:

% <<<<Малый правый поворот>>>>
\begin{center}
	\begin{tikzpicture}[sibling distance=4cm, level distance=1.8cm]
	\node [circle, minimum size=1.2cm, draw] {\large $A_{k+2}$}
		child {node [circle, minimum size=1.2cm, draw] {\large $B_{k+1}$}
			child {node [rectangle, minimum height=1.2cm, minimum width=1.2cm, draw] {\large $\alpha_{k}$}}
			child {node [rectangle, minimum height=1.2cm, minimum width=1.2cm, draw] {\large $\beta_{k}$}}
		}
		child {node [rectangle, minimum height=1.2cm, minimum width=1.2cm, draw] {\large $\gamma_{k}$}};
	\end{tikzpicture}
\end{center}
% <<<<Малый правый поворот>>>>

Высоты его поддеревьев таковы: $H(\alpha)=H(\beta)=H(\gamma)=k, H(A)=k+2, H(B)=k+1$. Условие баланса выполнено. Увеличим 
высоту поддерева $\alpha$, добавив элемент, и получим: $H(\alpha)=k+1, H(\beta)=k, H(\gamma)=k, H(A)=k+3, H(B)=k+2$. Дерево разбалансировано.

% <<<<Малый правый поворот>>>>
\begin{center}
	\begin{tikzpicture}[sibling distance=4cm, level distance=1.8cm]
	\node [circle, minimum size=1.2cm, draw] {\large $A_{k+3}$}
		child {node [circle, minimum size=1.2cm, draw] {\large $B_{k+2}$}
			child {node [rectangle, minimum height=1.2cm, minimum width=1.2cm, draw] {\large $\alpha_{k+1}$}}
			child {node [rectangle, minimum height=1.2cm, minimum width=1.2cm, draw] {\large $\beta_{k}$}}
		}
		child {node [rectangle, minimum height=1.2cm, minimum width=1.2cm, draw] {\large $\gamma_{k}$}};
	\end{tikzpicture}
\end{center}
% <<<<Малый правый поворот>>>>

Преобразуем это дерево малым правым поворотом:

% <<<<Малый правый поворот>>>>
\begin{center}
	\begin{tikzpicture}[sibling distance=4cm, level distance=1.8cm]
	\node [circle, minimum size=1.2cm, draw] {\large $B_{k+2}$}
		child {node [rectangle, minimum height=1.2cm, minimum width=1.2cm, draw] {\large $\alpha_{k+1}$}}
		child {node [circle, minimum size=1.2cm, draw] {\large $A_{k+1}$}
			child {node [rectangle, minimum height=1.2cm, minimum width=1.2cm, draw] {\large $\beta_{k}$}}
			child {node [rectangle, minimum height=1.2cm, minimum width=1.2cm, draw] {\large $\gamma_{k}$}}
		};
	\end{tikzpicture}
\end{center}
% <<<<Малый правый поворот>>>>

Переписав три указателя, мы сбалансировали это дерево (за $O(1)$). При этом сохранилось свойство поиска. Малый левый 
поворот выполняется аналогично, но только в инвертированной ситуации.

\newpage
\subsection{Большой правый и левый повороты}

Рассмотрим следующее АВЛ дерево:

% <<<<Большой правый поворот>>>>
\begin{center}
	\begin{tikzpicture}[sibling distance=4cm, level distance=1.8cm]
	\node [circle, minimum size=1.2cm, draw] {\large $A_{k+3}$}
		child {node [circle, minimum size=1.2cm, draw] {\large $B_{k+2}$}
			child {node [rectangle, minimum height=1.2cm, minimum width=1.2cm, draw] {\large $\alpha_{k+1}$}}
			child {node [circle, minimum size=1.2cm, draw] {\large $C_{k+1}$}
				child {node [rectangle, minimum height=1.2cm, minimum width=1.2cm, draw] {\large $\beta_{k}$}}
				child {node [rectangle, minimum height=1.2cm, minimum width=1.2cm, draw] {\large $\gamma_{k}$}}
			}
		}
		child {node [rectangle, minimum height=1.2cm, minimum width=1.2cm, draw] {\large $\delta_{k+1}$}};
	\end{tikzpicture}
\end{center}
% <<<<Большой правый поворот>>>>

Высоты его поддеревьев таковы: $H(\beta)=H(\gamma)=k, H(\alpha)=H(\delta)=k+1, H(A)=k+3, H(B)=k+2, H(C)=k+1$. Условие баланса выполнено. 
Увеличим высоту поддерева $\beta$, добавив элемент, и получим: $H(\alpha)=H(\delta)=H(\beta)=k+1, H(\gamma)=k, H(A)=k+4, H(B)=k+3, H(C)=k+2$. 
Дерево разбалансировано.

% <<<<Большой правый поворот>>>>
\begin{center}
	\begin{tikzpicture}[sibling distance=4cm, level distance=1.8cm]
	\node [circle, minimum size=1.2cm, draw] {\large $A_{k+4}$}
		child {node [circle, minimum size=1.2cm, draw] {\large $B_{k+3}$}
			child {node [rectangle, minimum height=1.2cm, minimum width=1.2cm, draw] {\large $\alpha_{k+1}$}}
			child {node [circle, minimum size=1.2cm, draw] {\large $C_{k+2}$}
				child {node [rectangle, minimum height=1.2cm, minimum width=1.2cm, draw] {\large $\beta_{k+1}$}}
				child {node [rectangle, minimum height=1.2cm, minimum width=1.2cm, draw] {\large $\gamma_{k}$}}
			}
		}
		child {node [rectangle, minimum height=1.2cm, minimum width=1.2cm, draw] {\large $\delta_{k+1}$}};
	\end{tikzpicture}
\end{center}
% <<<<Большой правый поворот>>>>

Преобразуем это дерево большим правым поворотом:

% <<<<Большой правый поворот>>>>
\begin{center}
	\begin{tikzpicture}[level/.style={sibling distance=5cm/#1}, level distance=1.8cm]
	\node [circle, minimum size=1.2cm, draw] {\large $C_{k+3}$}
		child {node [circle, minimum size=1.2cm, draw] {\large $B_{k+2}$}
			child {node [rectangle, minimum height=1.2cm, minimum width=1.2cm, draw] {\large $\alpha_{k+1}$}}
			child {node [rectangle, minimum height=1.2cm, minimum width=1.2cm, draw] {\large $\beta_{k+1}$}}
		}
		child {node [circle, minimum size=1.2cm, draw] {\large $A_{k+2}$}
			child {node [rectangle, minimum height=1.2cm, minimum width=1.2cm, draw] {\large $\gamma_{k}$}}
			child {node [rectangle, minimum height=1.2cm, minimum width=1.2cm, draw] {\large $\delta_{k+1}$}}
		};
	\end{tikzpicture}
\end{center}
% <<<<Большой правый поворот>>>>

Переписав пять указателей, мы сбалансировали это дерево (за $O(1)$). Так же сохранилось свойство поиска.
Большой левый поворот выполняется аналогично, но только в инвертированной ситуации.

Так же стоит отметить, что неважно, в какое поддерево вершины $C$ будет добавлен элемент. В любом случае большой поворот 
сбалансирует дерево. Несложно видеть, что большой правый поворот является композицией малого левого поворота относительно 
вершины $B$ и малого правого поворота относительно вершины $C$. Аналогично большой левый поворот является композицией
малого правого и малого левого поворотов.

\subsection{Добавление элемента}

Мы хотим вставлять элементы в АВЛ дерево за $O(\log n)$, поэтому нам нужно постоянно поддерживать его сбалансированным. 
Введем для каждой вершины величину баланса $balance(X) \in \{-1, 0, 1\}$, равную разности высот правого и левого 
поддеревьев этой вершины. Рассмотрим следующее АВЛ дерево:

% <<<<Вставка в АВЛ дерево>>>>
\begin{center}
	\begin{tikzpicture}[level/.style={sibling distance=5cm/#1}, level distance=1.8cm]
	\node [circle, minimum size=1.2cm, draw] {\large $A$}
		child {node [circle, minimum size=1.2cm, draw] {\large $B$}
			child {node [rectangle, minimum height=1.2cm, minimum width=1.2cm, draw] {\large $\alpha$}}
			child {node [rectangle, minimum height=1.2cm, minimum width=1.2cm, draw] {\large $\beta$}}
		}
		child {node [circle, minimum size=1.2cm, draw] {\large $C$}
			child {node [rectangle, minimum height=1.2cm, minimum width=1.2cm, draw] {\large $\gamma$}}
			child {node [rectangle, minimum height=1.2cm, minimum width=1.2cm, draw] {\large $\delta$}}
		};
	\end{tikzpicture}
\end{center}
% <<<<Вставка в АВЛ дерево>>>>

Будем добавлять элементы в левое и правое поддерево и смотреть, как меняются величины баланса для вершин $A$, $B$, $C$. 
Составим таблицу возможных изменений:

% <<<<Таблица балансов>>>>
\begin{center}
	\begin{tabular}{ |c | c | c | c | c | c | c | c | c | c | c |} \cline{1-5} \cline{7-11}
		\multirow{2}*{$Rotate$} & \multirow{2}*{$balance(B)$} & \multicolumn{3}{| c |}{$balance(A)$} & & \multirow{2}*{$Rotate$} & \multirow{2}*{$balance(C)$} & \multicolumn{3}{| c |}{$balance(A)$}\\ \cline {3-5} \cline{9-11}
		& & $-1$ & $0$ & $1$ & & & & $-1$ & $0$ & $1$ \\ \cline{1-5} \cline{7-11}
		$none$ & $1 \longrightarrow -1$ & $-1$ & $0$ & $1$ & & $none$ & $1 \longrightarrow -1$ & $-1$ & $0$ & $1$ \\ \cline{1-5} \cline{7-11}
		$none$ & $-1 \longrightarrow 0$ & $-1$ & $0$ & $1$ & & $none$ & $-1 \longrightarrow 0$ & $-1$ & $0$ & $1$ \\ \cline{1-5} \cline{7-11}
		$SRR$ & $0 \longrightarrow -1$ & \textbf{0*} & $-1$ & $0$ & & $BLR$ & $0 \longrightarrow -1$ & $0$ & $1$ & \textbf{0*} \\ \cline{1-5} \cline{7-11}
		$none$ & $0 \longrightarrow 0$ & $-1$ & $0$ & $1$ & & $none$ & $0 \longrightarrow 0$ & $-1$ & $0$ & $1$ \\ \cline{1-5} \cline{7-11}
		$BRR$ & $0 \longrightarrow 1$ & \textbf{0*} & $-1$ & $0$ & & $SLR$ & $0 \longrightarrow 1$ & $0$ & $1$ & \textbf{0*} \\ \cline{1-5} \cline{7-11}
		$none$ & $1 \longrightarrow 1$ & $-1$ & $0$ & $1$ & & $none$ & $1 \longrightarrow 1$ & $-1$ & $0$ & $1$ \\ \cline{1-5} \cline{7-11}
		$none$ & $1 \longrightarrow 0$ & $-1$ & $0$ & $1$ & & $none$ & $1 \longrightarrow 0$ & $-1$ & $0$ & $1$ \\ \cline{1-5} \cline{7-11}
	\end{tabular}
\end{center}
% <<<<Таблица балансов>>>>

Теперь, опираясь на данные таблицы, можно составить алгоритм вставки нового элемента в АВЛ дерево. Далее представлена 
процедура вставки, описанная на алгоритмическом языке (за основу взят С++).

\subsection{Процедура вставки}
\lstset{language = C++}
\begin{lstlisting}
void insert(Note* N, int x)
{
  int bal = 0;    // next vertex balance
  if (N != NULL)
    {
      if (N.value <= x)
      {
        bal = N.right*.bal;
        insert(N.right*, x);
        if (bal == 0 && N.right*.bal == 1 && N.bal == 1)
        {
          use SLR;
        }
        else
        if (bal == 0 && N.right*.bal == -1 && N.bal == 1)
        {
          use BLR;
        }
          update N.bal;
      }
      else
      {
        bal = N.left*.bal;
        insert(N.left*, x);
        if (bal == 0 && N.left*.bal == 1 && N.bal == -1)
        {
          use BRR;
        }
        else
        if (bal == 0 && N.left*.bal == -1 && N.bal == -1)
        {
          use SRR;
        }
        update N.bal;
      }
    }
    else
    {
      N = new Note;
      N.value = x;
      N.bal = 0;
    }
}
\end{lstlisting}

\end{document}