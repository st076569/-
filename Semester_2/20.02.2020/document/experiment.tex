% +------------------------------------------------------------------------------------------------------------------+
%   (15 баллов) Дан массив, содержащий перестановку чисел от 0 до N. Посчитать 
%   амортизационную стоимость линейного поиска элемента со значением i. 
%   Провести численный эксперимент.
% +------------------------------------------------------------------------------------------------------------------+

\documentclass[a4paper,12pt] {report} 			% размер бумаги устанавливаем А4, шрифт 12пунктов
\usepackage[utf8] {inputenc} 					% включаем свою кодировку: utf8 
\usepackage[english,russian] {babel} 			% используем русский и английский языки с переносами
\usepackage{misccorr}						% соответствие стандарту
\usepackage{listingsutf8}

\begin{document}

\setlength{\parskip}{4pt}

\begin{center}
	{\LARGE \bfseries \slshape Амортизационная стоимость поиска элемента в перестановке из N элементов}
\end{center}

\textit{
	Дан массив, содержащий перестановку чисел от 0 до N. Посчитать амортизационную стоимость линейного 
	поиска элемента со значением i. Провести численный эксперимент.
}

\section{Описание теории}

{\bfseries Амортизационный анализ} (англ. \textit{amortized analysis}) -- метод подсчёта времени, требуемого для выполнения последовательности операций над структурой данных. При этом время усредняется по всем выполняемым операциям, и анализируется средняя производительность операций в худшем случае.

{\bfseries Средняя амортизационная стоимость операций} -- величина $a$, находящаяся по формуле: $a = \frac{\sum\limits_{i=1}^{n} t_i}{n},$ где $t_1, t_2, \ldots, t_n$ -- время выполнения операций $1, 2, \ldots, n, $  совершённых над структурой данных.

Амортизационный анализ использует следующие методы:

\begin{enumerate}
	\item Метод усреднения
	\item Метод потенциалов
	\item Метод предоплаты
\end{enumerate}

Воспользуемся методом усреднения. В этом методе амортизационная стоимость операций определяется напрямую по формуле, указанной выше: суммарная стоимость всех операций алгоритма делится на их количество.

Пусть в перестановке $n$ элементов, тогда время поиска $i$-го элемента составит $O(n)$. Пусть было произведено $m$ операций, тогда: $$a = \frac{\sum\limits_{i=1}^{m} t_i}{m} = \frac{m \cdot O(n)}{m} = O(n),$$ где $t_i = O(n).$ Получили: $a = O(n)$

\section{Эксперимент}

Для проведения эксперимента была написана программа на языке С++ (текст программы расположен в файле \textbf{st076569\_7.cpp} в папке \textbf{source}). 
Она генерирует случайную перестановку из $n$ элементов. Далее $n$ раз случайным образом выбирает $i \in \{ 0, \ldots , n - 1\}$ (значение элемента, который будем искать) и осуществляет поиск, то есть считает кол-во шагов (каждый шаг выполняется за константное время) от начала массива до $i$. Все значения суммируются и делятся на $n$. Далее это повторяется еще для двух перестановок. В итоге получаем среднее время поиска элемента (в шагах) по трем случайным перестановкам из $n$ элементов. Обозначим это число как $average_n$.

Эти операции выполняются для $n \in \{200, 400, 600, \ldots , 4000\}$ и для каждого программа возвращает $average_n$. Например для $n = 600$ при нескольких запусках значения $average_{600}$ будут такими: $301.530$; $291.667$; $292.502$. Так же программа для каждого $n$ выводит коэффициент: $$k=\frac{average_n}{average_{200}}.$$ Он предназначен для анализа пропорциональности роста.

Далее производился анализ того, как меняется $average_n$ в зависимости от $n$. По теоретическим расчетам зависимость должна быть линейной.

\section{Результаты и выводы}

Программа запускалась 10 раз. Ее вывод перенаправлялся в файл \textbf{results.txt} в папке \textbf{binaries}. Результаты эксперимента показали, что зависимость времени поиска, выраженного в условных единицах, от количества элементов перестановки с некоторой погрешностью линейна.

Теоретические расчеты были подтверждены эксперементально, амортизационная стоимость линейного поиска элемента со значением $i$ в перестановке чисел от $0$ до  $n$ равна O(n).

\end{document}
