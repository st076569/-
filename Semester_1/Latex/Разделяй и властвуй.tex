\documentclass [a4paper,12pt] {report} 	% размер бумаги устанавливаем А4, шрифт 12пунктов
\usepackage [utf8] {inputenc} 				% включаем свою кодировку: utf8 
\usepackage [english,russian] {babel} 		% используем русский и английский языки с переносами
\usepackage {listings}						% для обработки кода С++
\usepackage {misccorr}					% соответствие стандарту
\usepackage{graphicx}
\setlength{\parindent}{5ex}				%абзацный отступ

\begin{document}

\begin{center}
{\Large \bfseries \slshape Принцип \glqq Разделяй и властвуй\grqq{} на примере сортировки слиянием и бинарного поиска}
\end{center}

Для начала стоит сказать о том, что такое метод «разделяй и властвуй»? Главная его идея – разбиение основной задачи на несколько более маленьких, пока эти подзадачи не станут элементарными. Такой метод сокращает время работы программы, что легко доказывается путем применения основной теоремы о рекуррентных соотношениях ({\slshape Master theorem}), что будет показано на следующей лекции.

Для начала рассмотрим бинарный (двоичный) поиск. В чем заключается задача? Есть некоторый элемент \glqq {\bfseries x}\grqq{}, есть некоторая последовательность $a_{1}, a_{2},\ldots, a_{n}$, в которой каждый следующий элемент (например) больше предыдущего или равен ему. Требуется определить, найдется ли в данной последовательности такой элемент, который по величине равен элементу \glqq {\bfseries x}\grqq{}. Перед нами задача поиска. Можно осуществить полный перебор, но это не нужно, так как есть более эффективный подход.

Возьмем произвольный элемент $a_{k}$ такой, что $1 \le k \le n$, и сравним его с \glqq {\bfseries x}\grqq{}. Есть всего три варианта:

\begin{enumerate}
  \item $x < a_{k}$. Значит, ищем в левой части: $\{\,a_{1},\ldots,a_{k-1}\,\}$
  \item $x = a_{k}$. Элемент найден
  \item $x > a_{k}$. Значит, ищем в правой части: $\{\,a_{k+1},\ldots,a_{n}\,\}$
\end{enumerate}

Получается, что мы в определенное количество раз сокращаем длину массива за каждое сравнение. Если никакой дополнительной информации о структуре массива нет, то имеет смысл уменьшать длину массива вдвое за каждое сравнение, то есть $k = [\frac {n}{2}]$.

Какова сложность бинарного поиска? $O(n)$, затем $O(\frac{n}{2})$, $O(\frac{n}{4})$, \ldots , $O(\frac{n}{2^{m}})$. Получается следующее: $\frac{n}{2^{m}}=1$; $n=2^{m}$; $m=\log_{2}n$. Это и есть сложность бинарного поиска $O(\log_{2}n)$

Теперь рассмотрим задачи сортировки, а именно о сортировку слиянием ({\slshape Merge sort}). Главная идея алгоритма это~--- деление исходного массива на несколько массивов меньшего размера (Чтобы потом их быстро отсортировать). Рассмотрим принцип работы алгоритма на двух массивах. Пусть у нас есть массив \glqq {\bfseries а}\grqq{} длины \glqq {\bfseries n}\grqq{} и массив \glqq {\bfseries b}\grqq{} длины \glqq {\bfseries m}\grqq{}, из них мы хотим получить массив c длины \glqq {\bfseries n+m}\grqq{}. При этом два этих массива уже отсортированы. Реализуем алгоритм слияния.

\newpage

\lstset {language = C++, frame = single}
\begin{lstlisting}
i = 0;
j = 0;
for (int k = 0; k < n + m; k++)
{
    c[k] = min(a[i],b[j]);
    if (a[i] <= b[j])
    {
        i++;
    }
    else
    {
        j++;
    }
}
\end{lstlisting}

На каждом шаге мы берем два элемента из разных массивов, меньший из них записываем в новый массив, изменяем счетчик у этого массива, а после повторяем шаг. Надо только следить за счетчиками \glqq {\bfseries i}\grqq{} и \glqq {\bfseries j}\grqq{}. Когда элементы в каком-то массиве закончатся, нужно прекратить выполнение программы – ведь если элементы закончились в массиве \glqq {\bfseries а}\grqq{}, к примеру, то оставшиеся элементы в массиве \glqq {\bfseries с}\grqq{} будут состоять из оставшихся элементов массива \glqq {\bfseries b}\grqq{}.

Данный метод можно применить для сортировки массива. Возьмем исходный, разобьем его на две части, чтобы потом \glqq слить\grqq{} их воедино. Но эти два подмассива должны быть отсортированы. Тогда каждый из этих подмассивов нужно разбить еще на две части. А потом еще и еще. До тех пор, пока задача сортировки не станет элементарной, то есть, когда массивы не будут состоять из одного или двух элементов. Именно это и называется сортировкой слиянием.
Какова же сложность такого алгоритма? На первом шаге у нас два массива длины $\frac{n}{2}$, то есть сложность $O(n)$; на втором шаге у нас уже четыре массива, длина каждого из которых $\frac{n}{4}$, то есть сложность $4 \cdot O(\frac{n}{4}) = O(n)$ и так далее. Получается, что у нас $\log_{2}n$ шагов, сложность каждого из которых $O(n)$, то есть общая сложность~--- $O(n \cdot \log_{2}n)$.

\begin{center}
\begin {tabular}{l l l l}
1) & Массивы: & $\{a_{1}, \ldots, a_{\frac{n}{2}}\}, \{a_{\frac{n}{2}}, \ldots, a_{n}\}$ & ($2 \cdot O(\frac{n}{2})$) \\
\multicolumn{4}{l}{$\cdots$} \\
$log_{2}n$) & Массивы: & $\{a_{1}\}$, \ldots, $\{a_{n}\}$ & ($n \cdot O(1)$) \\
\end{tabular}
\end{center}

Но при этом алгоритм сортировки слиянием неэффективен при массивах небольшой длины, так что обычно используют {\slshape timsort}~--- это сортировка представляет собой гибрид из сортировки слиянием и сортировки вставками. Основной способ – сортировка слиянием, но с некоторого момента будет использоваться сортировка вставками, что эффективно для массивов небольшой длины.

\end{document}