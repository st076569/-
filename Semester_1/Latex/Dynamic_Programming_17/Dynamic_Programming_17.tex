% +-----------------------------------------------------------------------------------------------------------+
%  Динамическое программирование: поиск наибольшей общей подпоследовательности, 
%  вычисления расстояния Левенштейна, дискретная задача о рюкзаке.                           
% +-----------------------------------------------------------------------------------------------------------+

\documentclass[a4paper,12pt] {report} 			% размер бумаги устанавливаем А4, шрифт 12пунктов
\usepackage[utf8] {inputenc} 					% включаем свою кодировку: utf8 
\usepackage[english,russian] {babel} 			% используем русский и английский языки с переносами
\usepackage{misccorr}						% соответствие стандарту
\usepackage{amssymb}
\usepackage{amsmath}						

\begin{document}

\begin{center}
	{\LARGE \bfseries \slshape Динамическое программирование (примеры)}
\end{center}

\section{Дискретная задача о рюкзаке}

Есть рюкзак вместимости W, есть $k$ предметов с некоторым 
объемом $w_{i}$ и стоимостью $c_{i}$, где $i$~--~номер предмета. 
Нужно найти все предметы такие, чтобы выполнялось условие.

$$max \sum c_{i} : \sum w_{i} < W$$.

Представим набор предметов следующим образом:

\begin{center}
	\begin{tabular}{ c c c }
		Предмет & Стоимость & Объем  \\
		$p_{1}$ & $c_{1}$ & $w_{1}$ \\
		$p_{2}$ & $c_{2}$ & $w_{1}$ \\
		$p_{3}$ & $c_{3}$ & $w_{1}$ \\
		\ldots & \ldots & \ldots \\
		$p_{k}$ & $c_{k}$ & $w_{k}$
	\end{tabular}
\end{center}

Составим таблицу, в столбцах которой будут всевозможные веса, а в строках 
будут первые $i$ предметов набора. Первую строку и первый столбец, исходя 
из логики задачи, можно сразу обнулить.

\begin{center}
	\begin{tabular}{ | c | c c c c c | }
		\hline
		Набор & 0 & 1 & 2 & $\cdots$ & W \\
		\hline
		$\{ \varnothing \}$ & 0 & 0 & 0 & $\cdots$ & 0 \\
		$\{ p_{1} \}$ & 0 &   &  &  &  \\
		$\{ p_{1},p_{2} \}$ & 0 &   &  &  &  \\
		$\cdots$ & \vdots &  &  &  & \\
		$\{ p_{1}, \ldots p_{k} \}$ & 0 &   &  &  &  \\
		\hline
	\end{tabular}
\end{center}

Для заполнения таблицы потребуется рекурентное соотношение. 
Составим его ($C_{ij}$~--~значение в $i$ строке и $j$ столбце таблицы).

\begin{center}
	\begin{equation*}
	C_{ij} =
		\begin{cases}
			max \{ c_{i} + C_{(i - 1)(j - w_{i})}; C_{(i - 1)j} \}, & \text{$i,j > 0$} \\
			0, & \text{$i \leq 0$} \\
			0, & \text{$j \leq 0$} \\
		\end{cases}
	\end{equation*}
\end{center}

Прямым или обратным ходом заполняем таблицу. Ответ к исходной задаче 
будет находится на пересечении $k$ столбца и $k$ строки (В последней клетке таблицы).

\begin{center}
	\begin{tabular}{ | c | c c c c c | }
		\hline
		Набор & 0 & 1 & 2 & $\cdots$ & W \\
		\hline
		$\{ \varnothing \}$ & 0 & 0 & 0 & $\cdots$ & 0 \\
		$\{ p_{1} \}$ & 0 & $C_{11}$ & $C_{12}$ & $\cdots$ & $C_{1W}$ \\
		$\{ p_{1},p_{2} \}$ & 0 & $C_{21}$  & $C_{22}$ & $\cdots$ & $C_{2W}$ \\
		$\cdots$ & \vdots & \vdots & \vdots &  & \vdots \\
		$\{ p_{1}, \ldots p_{k} \}$ & 0 & $C_{k1}$ & $C_{k2}$ & $\cdots$ & $C_{kW}$ \\
		\hline
	\end{tabular}
\end{center}

Сложность алгоритма: $O(kW)$. В данной задаче большую роль играют единицы измерения объема. 
В некоторых случаях динамическое решение может оказаться менее эффективным в сравнении с полным перебором 
вариантов. Так происходит, если объем не является целым числом. В этом случае нужно брать 
значение с некоторой погрешностью. И номера столбцов должны быть пропорциональны $gcd(w_{1}, \ldots , w_{k})$.

\section{Расстояние Левенштейна}

Расстояние Левенштейна (редакционное расстояние, дистанция редактирования)~--~Минимальное количество односимвольных 
операций (вставки, удаления, замены), необходимых для превращения одной строки в другую.
В общем случае, операциям, используемым в этом преобразовании, можно назначить разные цены. 

Например, возьмем строки: $ABC$ и $BCD$. Преобразуем первую из них.

$$ABC \rightarrow BC \rightarrow BCD$$

Расстояние Левенштейна для этих строк равно 2. Теперь рассмотрим 
общее динамическое решение задачи на примере. Даны следующие строки.

\begin{center}
	\begin{tabular}{ c c }
		Строка 1 ($S_{1}$) : & abcbc \\
		Строка 2 ($S_{2}$) : & ccba \\
	\end{tabular}
\end{center}

Будем рассматривать все подстроки этих строк. В столбцах таблицы будут расположены 
подстроки $S_{1}$, а в строках таблицы будут расположены подстроки $S_{2}$. Составим таблицу.

\begin{center}
	\begin{tabular}{ | c | c c c c c c | }
		\hline
		& $\{ \varnothing \}$ & a & ab & abc & abcb & abcbc \\
		\hline
		$\{ \varnothing \}$ & & & & & & \\
		c & & & & & & \\
		cc & & & & & & \\
		ccb & & & & & & \\
		ccba & & & & & & \\
		\hline
	\end{tabular}
\end{center}

Для заполнения таблицы потребуется рекурентное соотношение. Составим его 
($D_{ij}$~--~расстояние Левенштейна для подстрок в $i$ строке и $j$ столбце таблицы). 
Также $S_{1,2}(i)$ это $i$ символ строки.

\begin{center}
	\begin{equation*}
	D_{ij} =
		\begin{cases}
			min \{ D_{i(j - 1)} + 1; D_{(i - 1)j} + 1; D_{(i - 1)(j -1)} + 1*(S_{1}(j) \neq S_{2}(i)) \}, & \text{$i,j > 0$} \\
			0, & \text{$i = j = 0$} \\
			$i$, & \text{$j = 0$} \\
			$j$, & \text{$i = 0$} \\
		\end{cases}
	\end{equation*}
\end{center}

Тепрь заполним таблицу. Ответ на поставленную задачу будет 
находиться в последней клетке.

\begin{center}
	\begin{tabular}{ | c | c c c c c c | }
		\hline
		& $\{ \varnothing \}$ & a & ab & abc & abcb & abcbc \\
		\hline
		$\{ \varnothing \}$ & 0 & 1 & 2 & 3 & 4 & 5 \\
		c & 1 & 1 & 2 & 2 & 3 & 4 \\
		cc & 2 & 2 & 2 & 2 & 3 & 3 \\
		ccb & 3 & 3 & 2 & 3 & 2 & 3 \\
		ccba & 4 & 3 & 3 & 3 & 3 & 3\\
		\hline
	\end{tabular}
\end{center}

Получим следующее преобразование строки ccba:

\begin{center}
	ccba $\rightarrow$ accba $\rightarrow$ abcba $\rightarrow$ abcbc
\end{center}

В данной задаче можно также рассматривать операции (удаления, замены, добавления), у которых есть цена, вес. 
В этом случае алгоритм идентичный, но в таблицу теперь нужно записывать соответствующую стоимость операции.

\section{Поиск наибольшей общей подпоследовательности}

Рассмотрим общее динамическое решение задачи на примере. Даны следующие строки.

\begin{center}
	\begin{tabular}{ c c }
		Строка 1 ($S_{1}$) : & abcbc \\
		Строка 2 ($S_{2}$) : & ccba \\
	\end{tabular}
\end{center}

Требуется найти их наибольшую общую подпоследовательность. Составим таблицу по тому же 
принципу, что и в расстоянии Левенштейна. Рассмотрим все возможные подстроки.

\begin{center}
	\begin{tabular}{ | c | c c c c c c | }
		\hline
		& $\{ \varnothing \}$ & a & ab & abc & abcb & abcbc \\
		\hline
		$\{ \varnothing \}$ & & & & & & \\
		c & & & & & & \\
		cc & & & & & & \\
		ccb & & & & & & \\
		ccba & & & & & & \\
		\hline
	\end{tabular}
\end{center}

Для заполнения таблицы потребуется рекурентное соотношение. Составим его 
($lcs_{ij}$~--~длина наибольшей общей подпоследовательности подстрок в $i$ строке и $j$ столбце таблицы). 
Также $S_{1,2}(i)$ это $i$ символ строки.

\begin{center}
	\begin{equation*}
	lcs_{ij} =
		\begin{cases}
			max \{ lcs_{i(j - 1)}; lcs_{(i - 1)j}; lcs_{(i - 1)(j -1)} + 1*(S_{1}(j) = S_{2}(i)) \}, & \text{$i,j > 0$} \\
			0, & \text{$i = 0$ or $j = 0$} \\
		\end{cases}
	\end{equation*}
\end{center}

Тепрь заполним таблицу. Ответ на поставленную задачу будет находиться в последней клетке.

\begin{center}
	\begin{tabular}{ | c | c c c c c c | }
		\hline
		& $\{ \varnothing \}$ & a & ab & abc & abcb & abcbc \\
		\hline
		$\{ \varnothing \}$ & 0 & 0 & 0 & 0 & 0 & 0 \\
		c & 0 & 0 & 0 & 1 & 1 & 1 \\
		cc & 0 & 0 & 0 & 1 & 1 & 2 \\
		ccb & 0 & 0 & 1 & 1 & 2 & 2 \\
		ccba & 0 & 1 & 1 & 1 & 2 & 2\\
		\hline
	\end{tabular}
\end{center}

Получили, что cb~--~наибольшая общая подпоследовательность строк ccba и abcbc.

\end{document}