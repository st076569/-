\documentclass [a4paper,12pt] {report} 		% размер бумаги устанавливаем А4, шрифт 12пунктов
\usepackage [utf8] {inputenc} 					% включаем свою кодировку: utf8 
\usepackage [english,russian] {babel} 			% используем русский и английский языки с переносами
\usepackage {misccorr}							% соответствие стандарту

\begin {document}
{\bfseries Решение задач. Асимптотические оценки.}

Упорядочите функции по возрастанию скорости роста (сверху — медленнее всего растущая функция, снизу — быстрее всего растущая).

\begin {enumerate}
  \item $\log_{2}\log_{2}n$
  \item $\sqrt{\log_{4}n}$
  \item $\log_{3}n$
  \item $(\log_{2}n)^2$
  \item $\sqrt{n}$
  \item $\frac{n}{\log_{5}n}$
  \item $\log_{2}(n!)$
  \item $3^{\log_{2}n}$
  \item $n^2$
  \item $7^{\log_{2}n}$
  \item $(\log_{2}n)^{\log_{2}n}$
  \item $n^{\log_{2}n}$
  \item $n^{\sqrt{n}}$
  \item $2^n$
  \item $4^n$
  \item $2^{3n}$
  \item $n!$
  \item $2^{2^n}$
\end {enumerate}

{\bfseries Пояснения:}

Функцию $2^{2^n}$ можно представить как: $e^{{2^n} \cdot \ln2}$, а функцию $n!$ как: $e^{\ln{n!}}$, где $\ln{n!} \approx n \cdot \ln{n} - n$. Видно, что $n \cdot \ln{n} - n$ растет медленнее, чем ${2^n} \cdot \ln2$, а значит и $n!$ растет медленнее, чем $2^{2^n}$.

При этом $n!$ растет быстрее, чем $2^{3n}$; $4^n$; $2^n$, так как факториал растет быстрее любой показательной функции. $2^n$; $4^n$; $2^{3n}$ расположены по порядку возрастания скорости роста так, как $2 < 4 < 2^3$.

Теперь возьмем функции $2^n$; $n^{\sqrt{n}}$; $n^{\log_{2}n}$ и представим их так: $e^{n \cdot \ln2}$; $e^{\sqrt{n} \cdot \ln{n}}$; $e^{\log_{2}(n) \cdot \ln{n}}$. Видно, что $n$ растет быстрее, чем $\sqrt{n} \cdot \ln{n}$ (так как $\ln{n}$ растет с меньшей скоростью, чем $\sqrt{n}$). Значит $2^n$ будет расти быстрее, чем $n^{\sqrt{n}}$. А $n^{\log_{2}n}$ будет расти медленнее, чем $n^{\sqrt{n}}$ (так как $\log_{2}n$ растет медленнее $\sqrt{n}$).

$(\log_{2}n)^{\log_{2}n} = o(n^{\log_{2}n})$, так как $\log_{2}n = o(n)$.

$7^{\log_{2}n} = o((\log_{2}n)^{\log_{2}n})$, так как $7 = o(\log_{2}n)$. 

$n^2 = o(7^{\log_{2}n})$, представим эти функции так: $(e^{2})^{\ln{n}}$; $(e^{\frac {\ln{7}}{\ln2}})^{\ln{n}}$.
Последняя возрастает быстрее, так как $e^{\frac {\ln{7}}{\ln2}} > e^2$. Но $3^{\log_{2}n} = o(n^2)$, так как $e^{\frac {\ln{3}}{\ln2}} < e^2$.

$\log_{2}(n!) = o(3^{\log_{2}n})$, представим эти функции так: $\frac {n \cdot \ln{n} - n}{\ln2}$; $n^{\log_{2}3}$. Видно, что $n^{\log_{2}3}$ возрастает быстрее, чем $n^{1,5}$, которая возрастает быстрее функции $n \cdot \ln{n}$. Значит $ \frac {n \cdot \ln{n} - n}{\ln2} = o(n^{\log_{2}3})$.

$\frac{n}{\log_{5}n} = o(\log_{2}(n!))$,  представим эти функции так: $e^{\ln{n}-\ln{\log_{5}n}}$; $e^{\frac {n \cdot \ln{n} - n}{\ln2}}$. Видно, что $\ln{n}-\ln{log_{5}n} = o(n \cdot \ln{n} - n)$.

$\sqrt{n} = o(\frac{n}{\log_{5}n})$, представим эти функции так: $(e^{\frac{1}{2}})^{\ln{n}}$; $e^{\frac {n \cdot \ln{n} - n}{\ln2}}$. Видно, что $\ln{n} = o(n \cdot \ln{n})$. Значит оценка правильная.

$(\log_{2}n)^2 = o(\sqrt{n})$, представим эти функции так: $e^{2 \cdot \ln{log_{2}n}}$; $e^{\frac{1}{2} \cdot \ln{n}}$. Видно, что $2 \cdot \ln{log_{2}n} = o({\frac{1}{2}} \cdot {\ln{n}})$. Значит оценка правильная.

$\log_{3}n = o((\log_{2}n)^2)$, представим эти функции так: $e^{\ln{\log_{3}n}}$; $e^{2 \cdot \ln{\log_{2}n}}$. Видно, что $e^2 > e$. Значит вторая функция растет быстрее, чем первая.

Далее порядок сортировки функций по возрастанию очевиден.

\end {document}