%%%%%%%%%%%%%%%%%%%%%%%%%%%%%%%%%%%%%%%%%%%%%%%%%%%%%%%%%%%%%%%%%%%%%%%%%%%%%
% Баталов Семен, 2021                                                       %
%%%%%%%%%%%%%%%%%%%%%%%%%%%%%%%%%%%%%%%%%%%%%%%%%%%%%%%%%%%%%%%%%%%%%%%%%%%%%

\documentclass[12pt, a4paper]{article}
\usepackage[left=2.5cm, right=2.5cm, top=2.5cm, bottom=2.5cm]{geometry}
\usepackage[utf8]{inputenc}
\usepackage{graphicx}
\graphicspath{{./pictures/}}
\usepackage[english, russian]{babel}
\usepackage{indentfirst}
\usepackage{misccorr}
\usepackage{amsmath}

\title{Кластеризация точек на плоскости}
\author{Баталов Семен}
\date{25.02.2021}

\begin{document}
    
    \sloppy
    
    \maketitle
    
    \section{Постановка задачи}
    
    Рассматривается плоскость, на которую случайным образом наносятся точки. 
    Нужно решить задачу кластеризации методом <<\textbf{K-Means++}>>. 
    Используется язык <<\textbf{Python}>>, подробнее о программе можно узнать в 
    папке <<\textbf{source}>> проекта.
    
    \subsection{K-Means и K-Means++}
    
    Алгоритм <<\textbf{K-Means}>> разбивает множество элементов векторного 
    пространства на заранее известное число кластеров <<k>>. В нашем случае 
    размерность пространства равна 2, а элементами являются точки, которые 
    описываются только двумя параметрами: абсциссой и ординатой.
    
    Основная идея заключается в том, что на каждой итерации перевычисляется центр 
    масс для каждого кластера, полученного на предыдущем шаге, затем векторы 
    разбиваются на кластеры вновь в соответствии с тем, какой из новых центров 
    оказался ближе по выбранной метрике.
    
    Алгоритм завершается, когда на какой-то итерации не происходит изменения 
    внутрикластерного расстояния. Это происходит за конечное число итераций, так 
    как количество возможных разбиений конечного множества конечно, а на каждом 
    шаге суммарное квадратичное отклонение уменьшается, поэтому зацикливание 
    невозможно.
    
    Важным этапом в алгоритме является первоначальная инициализация центров 
    классов. В <<\textbf{K-Means}>> центры выбираются случайно.
    
    Алгоритм <<\textbf{K-Means++}>> ничем, кроме способа начальной инициализации 
    центров, не отличается от <<\textbf{K-Means}>>. В <<\textbf{K-Means++}>> 
    центры выбираются (как правило) удаленными друг от друга, что с большей 
    вероятностью приводит к лучшим результатам, чем случайная инициализация.
    
    \section{Инструменты}
    
    Для работы была выбрана библиотека <<\textbf{sklearn}>>, из нее были взяты 
    алгоритмы <<\textbf{KMeans}>> и <<\textbf{kmeans\_plusplus}>>. Первый из них 
    осуществляет метод <<\textbf{K-Means}>>, второй генерирует центры кластеров в 
    соответствии с алгоритмом <<\textbf{K-Means++}>>.
    
    Для генерации случайных точек на плоскости использовались алгоритмы 
    библиотеки <<\textbf{sklearn}>>, а именно <<\textbf{make\_blobs}>>. Этот 
    алгоритм создает многоклассовые наборы данных, выделяя каждому классу один 
    или несколько нормально распределенных кластеров точек. 
    <<\textbf{make\_blobs}>> обеспечивает контроль относительно центров и 
    стандартных отклонений каждого кластера и используется для тестирования алгоритмов кластеризации.
    
    \section{Результаты}
    
    В экспериментах варьировалось количество точек на плоскости, количество кластеров, коэффициенты, отвечающие за распределение точек по плоскости. 
    
    Основной задачей было показать конечный результат работы кластеризатора и в случае <<\textbf{K-Means++}>> оценить расположение классов, используя начальную инициализацию центров, сравнить результаты с оригинальным разбиением на классы.
    
    \begin{figure} [h]
        \center{\includegraphics[width = \textwidth]{9_300_2_analysis.png}}
        \caption{9 кластеров, 300000 точек}
        \label{image1}
    \end{figure}
    
    \begin{figure} [h]
        \center{\includegraphics[width = \textwidth]{7_500_2_analysis.png}}
        \caption{7 кластеров, 500000 точек}
        \label{image2}
    \end{figure}
    
    \begin{figure} [h]
        \center{\includegraphics[width = \textwidth]{5_800_2_analysis.png}}
        \caption{5 кластеров, 800000 точек}
        \label{image3}
    \end{figure}
    
    \begin{figure} [h]
        \center{\includegraphics[width = \textwidth]{5_400_3_analysis.png}}
        \caption{5 кластеров, 400000 точек}
        \label{image4}
    \end{figure}
    
    \begin{figure} [h]
        \center{\includegraphics[width = \textwidth]{12_400_3_analysis.png}}
        \caption{12 кластеров, 400000 точек}
        \label{image5}
    \end{figure}
    
\end{document}